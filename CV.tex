% !TEX TS-program = xelatex
% !TEX encoding = UTF-8 Unicode
% -*- coding: UTF-8; -*-
% vim: set fenc=utf-8

%%%%%%%%%%%%%%%%%%%%%%%%%%%%%%%%%%%%%%%%%%%%%%%%%%%%%%%%%%%%%%%%%
%% SIMPLE-RESUME-CV
%% <https://github.com/zachscrivena/simple-resume-cv>
%% This is free and unencumbered software released into the
%% public domain; see <http://unlicense.org> for details.
%%%%%%%%%%%%%%%%%%%%%%%%%%%%%%%%%%%%%%%%%%%%%%%%%%%%%%%%%%%%%%%%%

% See "README.md" for instructions on compiling this document.

\documentclass[letterpaper,MMMyyyy,nonstopmode]{simpleresumecv}
% Class options:
% a4paper, letterpaper, nonstopmode, draftmode
% MMMyyyy, ddMMMyyyy, MMMMyyyy, ddMMMMyyyy, yyyyMMdd, yyyyMM, yyyy

%%%%%%%%%%%%%%%%%%%%%%%%%%%%%%%%%%%%%%%%%%%%%%%%%%%%%%%%%%%%%%%%%
%% PREAMBLE.
%%%%%%%%%%%%%%%%%%%%%%%%%%%%%%%%%%%%%%%%%%%%%%%%%%%%%%%%%%%%%%%%%

% CV Info (to be customized).
\newcommand{\CVAuthor}{Brian W. Duff}
\newcommand{\CVTitle}{Brian W. Duff's Resume}
\newcommand{\CVNote}{Resume compiled on {\today}}
\newcommand{\CVWebpage}{http://www.brianwduff.com}

% PDF settings and properties.
\hypersetup{
pdftitle={\CVTitle},
pdfauthor={\CVAuthor},
pdfsubject={\CVWebpage},
pdfcreator={XeLaTeX},
pdfproducer={},
pdfkeywords={},
unicode=true,
bookmarks=true,
bookmarksopen=true,
pdfstartview=FitH,
pdfpagelayout=OneColumn,
pdfpagemode=UseOutlines,
hidelinks,
breaklinks}

\linespread{1.15}

% Shorthand.
\newcommand{\Code}[1]{\mbox{\textbf{#1}}}
\newcommand{\CodeCommand}[1]{\mbox{\textbf{\textbackslash{#1}}}}

%%%%%%%%%%%%%%%%%%%%%%%%%%%%%%%%%%%%%%%%%%%%%%%%%%%%%%%%%%%%%%%%%
%% ACTUAL DOCUMENT.
%%%%%%%%%%%%%%%%%%%%%%%%%%%%%%%%%%%%%%%%%%%%%%%%%%%%%%%%%%%%%%%%%

\begin{document}

%%%%%%%%%%%%%%%
% TITLE BLOCK %
%%%%%%%%%%%%%%%

\Title{\CVAuthor}

\begin{SubTitle}
5824 Northumberland Street, Pittsburgh, PA 15217
\par
\href{mailto:brian.w.duff@gmail.com}
{brian.w.duff@gmail.com}
\,\SubBulletSymbol\,
(585)\,747-8345
\end{SubTitle}

\begin{Body}

%%%%%%%%%%%%%%
%% CAREER SUMMARY %%
%%%%%%%%%%%%%%

\Section
{Summary}
{Summary}
{PDF:Summary}

\Entry
Motivated \textbf{software engineer} and dedicated team member seeking to solve \textbf{complex problems} using self-reliance, versatility, and research experience developed over 4 years at a startup. 


%%%%%%%%%%%%%%%
%% EDUCATION %%
%%%%%%%%%%%%%%%

\Section
{Education}
{Education}
{PDF:Education}

\Entry
\textbf{Carnegie Mellon University},
Pittsburgh, Pennsylvania
\hfill
\DatestampYMD{2009}{08}{15} --
\DatestampYMD{2012}{12}{15}


\Gap
\BulletItem
B.S. in
Electrical and Computer Engineering
\begin{Detail}
\SubBulletItem
Cumulative GPA: 3.6 / 4.0
\end{Detail}



%%%%%%%%%%%%%%%%%%%%%%%%%%%
%% EXPERIENCE %%
%%%%%%%%%%%%%%%%%%%%%%%%%%%

\Section
{Experience}
{Experience}
{PDF:Experience}

\Entry
Senior Software Engineer,
\textbf{SpiralGen, Inc.}
\hfill
\DatestampYMD{2013}{01}{28} --
\DatestampYMD{2017}{06}{31}

\BigGap
Reseached, developed, and commercialized the high-performance code-generation tool Spiral.

\BigGap

\BulletItem
DARPA HACMS (High-Assurance Cyber Military Systems)
\SubBulletItem
Researched and developed formally-verified software components for cyber-physical systems.
\SubBulletItem
Worked with air and ground, manual and autonomous robots and virtual robots.
\SubBulletItem
Released a commercial version of Spiral at IEEE High Performance Extreme Computing Conference.

\BulletItem
Code generation toolbox for ADAS (Advanced Driver-Assistance Systems)
\SubBulletItem
Developed a protype Matlab/Simulink toolbox interface for ADAS control code generation.

\BulletItem
DARPA BRASS (Building Resource Adaptive Software Systems)
\SubBulletItem
Developed software which can autonomously adapt to changes in hardware or libraries.
\SubBulletItem
Tested resource adaptation with synthetic apperature radar (SAR) test case.

\BulletItem
High Performance FFT library for Argonne National Laboratory (ANL)
\SubBulletItem
Created, using Spiral, a library of high-performance threaded batch FFTs for use on a supercomputer.

\BulletItem
Spiral Cloud Interface
\SubBulletItem
Developed an online code-generation system using Codebox, Amazon Web Services, and Docker.


\BigGap
\Entry
Summer Intern,
\textbf{General Motors Fuel Cell Activities}
\hfill
Summers 2011, 2012

\Gap
\BulletItem
Operated as interim software build manager with software design team using Simulink code generation.
\BulletItem
Researched and applied signal processing techniques to hydrogen fuel cells, scripted analysis in Matlab/Simulink.


%%%%%%%%%%%%%%%%%%%%%%%%%
%% RESEARCH EXPERIENCE %%
%%%%%%%%%%%%%%%%%%%%%%%%%
%
%\Section
%{Research Experience}
%{Research Experience}
%{PDF:ResearchExperience}
%
%\Entry
%\href{http://www.example.com/my-institute}
%{\textbf{Institute for Advanced Research}},
%Science College
%
%\Gap
%\BulletItem
%Undergraduate Research Student, Science Department
%\hfill
%\DatestampYMD{2004}{05}{15} --
%\DatestampYMD{2005}{05}{15}
%\begin{Detail}
%\SubBulletItem
%Project:
%Investigations on the Use of Lasers to Measure Climate Change
%\SubBulletItem
%Supervisors:
%Prof.~Jane~Citizen and
%Dr~Ann~Yone
%\SubBulletItem
%Focus:
%Climate change, lasers, statistical analysis, data analytics.
%\end{Detail}



%%%%%%%%%%%%%%%%%%
%% PUBLICATIONS %%
%%%%%%%%%%%%%%%%%%

\Section
{Publications}
{Publications}
{PDF:Publications}

% Declare a new group to limit the scope of \MaxNumberedItem to this subsection.
\begingroup

\Gap
B. Duff, D. Popovici, T. M. Low, J. Larkin, M. Franusich, F. Franchetti,
``Quantifying Performance Improvements of Ported Software Using Spiral,''
\textit{2017 IEEE High Performance Extreme Computing Conference Proceedings}

\Gap
DOE SBIR Phase I Proposal (awarded) coauthor, ``Security Hardened Cyber Components for Nuclear Power Plants"


\endgroup

%%%%%%%%%%%%%%%%%%
%% PRESENTATIONS %%
%%%%%%%%%%%%%%%%%%

\Section
{Presentations}
{Presentations}
{PDF:Presentations}

% Declare a new group to limit the scope of \MaxNumberedItem to this subsection.
\begingroup

B. Duff (speaker), J. Larkin, M. Franusich, F. Franchetti,
``Automatic Generation of 3D FFTs,"
\textit{2014 Oil \& Gas HPC Workshop}

\Gap
Demonstration of CMU team DARPA HACMS capabilities to Pentagon officials,
\textit{DARPA I2O Demo Day 2014}
\endgroup


%%%%%%%%%%%%%%%%%%%%
%% LANGUAGES %%
%%%%%%%%%%%%%

\Section
{Languages}
{Languages}
{PDF:Languages}

\begingroup
C, C++, Java, Python, Matlab, Simulink, JavaScript, GAP
\endgroup

%%%%%%%%%%%%%%%%%
%% SKILLS %%
%%%%%%%%

\Section
{Technical Skills}
{Technical Skills}
{PDF:TechnicalSkills}

\begingroup
Embedded Systems, High Performance Computing (HPC), Git, Subversion, Linux, Windows, Technical Writing, Controls, Signal Processing, Eclipse RCP, Make, NSIS 
\endgroup


\end{Body}
\end{document}
